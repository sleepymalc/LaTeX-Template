\documentclass[13pt]{article}
\usepackage{amsmath}
\usepackage{geometry}
\usepackage{booktabs}
\usepackage{mathtools}
\usepackage{enumerate}
\usepackage{amssymb}
\usepackage{graphicx}
\usepackage{subfigure}
\usepackage{upgreek}
\usepackage{multirow}
\usepackage{indentfirst}
\usepackage{color}
\usepackage{bm}
\usepackage{float}
\usepackage{caption}
\usepackage{amstext}
\usepackage{array}
\usepackage{enumerate}
\usepackage{subfigure}
\usepackage{xcolor}
\usepackage{pdfpages}
\usepackage{bm}
\usepackage{tikz}
\usepackage{etoolbox} % Required for if statements
\usepackage{mathrsfs}
\usepackage{physics}
\usepackage{siunitx}
\usepackage{pgfplots}
\usepackage{pdfpages}
\geometry{a4paper,left=2cm,right=2cm,top=2cm,bottom=2cm}
\thispagestyle{empty}

\definecolor{Turquoise3}{RGB}{0, 134, 139}
\renewcommand{\emph}[1]{{\color{Turquoise3}\textsl{#1}}}
\newcommand{\C}{\mathbb{C}} \newcommand{\F}{\mathbb{F}} \newcommand{\R}{\mathbb{R}} \newcommand{\Q}{\mathbb{Q}} \newcommand{\N}{\mathbb{N}} \newcommand{\Z}{\mathbb{Z}}
\newcommand{\myqed}{\hfill$\blacksquare$}
\newcommand{\nullspacesmall}{~\\[2pt]}
\newcommand{\nullspacemid}{~\\[8pt]}
\newcommand{\nullspacebig}{~\\[12pt]}
\newcommand{\at}[3]{\left.#1\right\vert_{#2}^{#3}}

\begin{document}
    \vspace{5cm}
    \begin{center}
        \rule{15cm}{0.01cm}
        \\\LARGE{
            UM-SJTU Joint Institute
            \\Summer 2021
            \\Probabilistic Methods in Engineering
            \\Ve401
          }
        \\\rule{15cm}{0.01cm}
        \\\vspace{6cm}
        \begin{Huge}
            \sc{Assignment 9}
        \end{Huge}
    \end{center}
    \vfill
    \flushleft
    \begin{center}
        \sc{Assignment Group 31}
        \begin{align*}
            \qquad\qquad\text{Name: }   \qquad& \text{Pingbang Hu}  &   \text{ID:} \qquad \text{519370910026}\qquad\qquad\\
            \qquad\qquad\text{Name: }   \qquad& \text{Jeongsoo Pang}&   \text{ID:} \qquad \text{519370990016}\qquad\qquad\\ 
            \qquad\qquad\text{Name: }   \qquad& \text{Zihao Xu}     &   \text{ID:} \qquad \text{519370910174}\qquad\qquad
        \end{align*}
        \mbox{}
        \\\sc{Date: 17 July 2021}
    \end{center}

\setlength{\parindent}{1em}
\newpage
\thispagestyle{empty}
\setcounter{page}{1}

\newpage
%------------------------------------------------Exercise 9.1----------------------------------------------------------------------------------------
\section*{Exercise 9.1}
%------------------------------------------------Exercise 9.1 i--------------------------------------------------------------------------------------
\subsection*{(i)}
\par We set up the null hypothesis as 
\begin{equation*}
    H_0 : \mu_{PCB} \geq \mu_{\neg PCB}
\end{equation*}
where $\mu_{PCB}$ denotes the case that birds exposed to PCB, while $\mu_{\neq PCB}$ denotes the case that 
birds does not expose to PCB.

\myqed

%------------------------------------------------Exercise 9.1 ii--------------------------------------------------------------------------------------
\subsection*{(ii)}
\par Firstly, Student's T test is not suitable since we do not know whether the exposure to PCB will affect the variance of 
the distribution of the thickness of eggs or not. And to use Student's t test, we need the condition of \emph{equal variance}.

\par Secondly, assuming the thickness is normal distributed regardless of the exposure of PCB is reasonable. Hence, we do not have 
a strong reason not to use Welch's T test but use Wilcoxon Rank-Sum test.

\par Finally, since Wilcoxon Rank-Sum test only compare the \emph{locations} of two distribution, which is not exactly what we want.
Therefore, we choose to use \textbf{Welch's T test}.

\myqed

%------------------------------------------------Exercise 9.1 iii------------------------------------------------------------------------------------
\subsection*{(iii)}
\par We set up the hull hypothesis as in (i) and perform a Fisher test. We first note that $H_0$ is equivalent to 
\begin{equation*}
    H_0 : \mu_{PCB} - \mu_{\neg PCB} \geq 0
\end{equation*}

\par By {\tt Mathematica}, we have 
\begin{equation*}
    T_{\gamma} = -0.956
\end{equation*}
with 
\begin{equation*}
    \gamma = 15.99 \approx 16
\end{equation*}
degrees of freedom.

\par Furthermore, at significance level of $95\%$, we have 
\begin{equation*}
    t_{0.95, 16} = 1.745.
\end{equation*}

\par We then have 
\begin{equation*}
    T_\gamma > -t_{0.95, \gamma}
\end{equation*}
which implies that we do not have enough evidence to reject $H_0$. 

\par Hence, we conclude that there is no evidence that exposed to PCB will have a negative effect on the reproductive 
ability of scree h owls.

\myqed

\newpage
%------------------------------------------------Exercise 9.2----------------------------------------------------------------------------------------
\section*{Exercise 9.2}
\subsection*{Paired $T$-test}
\par We set up the null hypothesis as 
\begin{equation*}
    H_0 : \mu_{X_V} \geq \mu_{X_A}
\end{equation*}
where $X_V$ and $X_A$ are the reaction time for visual and auditory respectively. This is equivalent as 
\begin{equation*}
    H_0 : \mu_D \geq 0
\end{equation*}
where $D := X_V - X_A$

\par We assume that $X_V$ and $X_A$ follow a joint bivariate normal distribution. Then $D$ follows a normal distribution.
Now, we have 
\begin{table}[H]
    \centering
    \begin{tabular}{c | ccccccccccccccc}
        Subject & 1 & 2 & 3 & 4 & 5 & 6 & 7 & 8 & 9 & 10 & 11 & 12 & 13 & 14 & 15\\
        \toprule
        $X_V$           & 161   & 203   & 235   & 176   & 201   & 188   & 228   & 211   & 191   & 178   & 159   & 227   & 193   & 192   & 212   \\
        $X_A$           & 157   & 207   & 198   & 161   & 234   & 197   & 180   & 165   & 202   & 193   & 173   & 137   & 182   & 159   & 156   \\
        $D = X_V - X_A$ & 4     & -4    & 37    & 15    & -33   & -9    & 48    & 46    & -11   & -15   & -14   & 90    & 11    & 33    & 56    \\
    \end{tabular}
\end{table}

\par We have
\begin{equation*}
    T_{14} = \at{\frac{\overline{D} - \mu_D}{\sqrt{S_D^2/n}}}{\mu_D = 0}{} = 1.946
\end{equation*}

\par For a $95\%$ of confidence level, since $t_{0.05, 14} = 1.761$, we have no strong evidence to reject $H_0$ at the $0.05$ level 
of significance. But nevertheless, we see that our test statistic is not far from $t_{0.05, 14}$, hence it is possible that the power is 
not enough for us to reject $H_0$ because of the small sample size.
 
\myqed

\newpage
\subsection*{Wilcoxon signed rank test}
\par We set up the null hypothesis as 
\begin{equation*}
    H_0 : P[X_{V} < X_{A}] \leq \frac{1}{2}
\end{equation*}
where $X_V$ and $X_A$ are the reaction time for visual and auditory respectively. This null hypothesis can be interpreted as 
"the visual reaction time is quicker than the auditory reaction time."

\par By using {\tt Mathematica}, we have the following rank list
\begin{center}
$\begin{array}{ccc | ccc}
    $\SI{}{ms}$ &   \text{Subject}  &   \text{Rank} &   $\SI{}{ms}$ &   \text{Subject}  &   \text{Rank} \\
    \toprule
    137         &   \text{Auditory} &   1           &   192         &   \text{Visual}   &   16          \\
    156         &   \text{Auditory} &   2           &   193         &   \text{Auditory} &   17.5        \\
    157         &   \text{Auditory} &   3           &   193         &   \text{Visual}   &   17.5        \\
    159         &   \text{Auditory} &   4.5         &   197         &   \text{Auditory} &   19          \\
    159         &   \text{Visual}   &   4.5         &   198         &   \text{Auditory} &   20          \\
    161         &   \text{Auditory} &   6.5         &   201         &   \text{Visual}   &   21          \\
    161         &   \text{Visual}   &   6.5         &   202         &   \text{Auditory} &   22          \\
    165         &   \text{Auditory} &   8           &   203         &   \text{Visual}   &   23          \\
    173         &   \text{Auditory} &   9           &   207         &   \text{Auditory} &   24          \\
    176         &   \text{Visual}   &   10          &   211         &   \text{Visual}   &   25          \\
    178         &   \text{Visual}   &   11          &   212         &   \text{Visual}   &   26          \\
    180         &   \text{Auditory} &   12          &   227         &   \text{Visual}   &   27          \\
    182         &   \text{Auditory} &   13          &   228         &   \text{Visual}   &   28          \\
    188         &   \text{Visual}   &   14          &   234         &   \text{Auditory} &   29          \\
    191         &   \text{Visual}   &   15          &   235         &   \text{Visual}   &   30          \\
\end{array}$
\end{center}

\par Since the two sample size are equal(both are $15$), we randomly choose the subject of \textbf{Auditory} and get the 
sum of the ranks as 
\begin{equation*}
    w_{15} = 1+2+3+4.5+6.5+8+9+12+13+17.5+19+20+22+24+29 = 190.5
\end{equation*}

\par Given the large sample sizes, we use the normal approximation for the test statistic. We have 
\begin{equation*}
    \begin{split}
        E[W_{15}] &= \frac{15(15 + 15 + 1)}{2} = \frac{465}{2}\\
        \text{Var}[W_{15}] &= \frac{15\cdot 15(15 + 15 + 1)}{12 - \sum\limits_{\text{groups}}\at{\frac{t^3 + t}{12}}{t = 3}{}} = \frac{13950}{19}\\
    \end{split}
\end{equation*}
where the sum is taken over all groups of $t$ ties, and in this case, $3$ ties occur.

\par Therefore, we know that 
\begin{equation*}
    Z = \frac{W_{15} - \frac{465}{2}}{\sqrt{\frac{13950}{19}}}
\end{equation*}
follows a standard normal distribution if $P[X_V < X_A] = \frac{1}{2}$. Then the value of our test statistic is 
\begin{equation*}
    z = \frac{190.5 - \frac{465}{2}}{\sqrt{\frac{13950}{19}}} = -1.550
\end{equation*}

\par Hence, by using {\tt Mathematica}, we find a $P-$value of 
\begin{equation*}
    P[Z\geq -1.550] = 0.939432
\end{equation*}

\par We see that there may be a small indication for us to reject $H_0$, namely a small evidence that the visual reaction time 
tends to be slower than the auditory reaction time.

\myqed

\newpage
%------------------------------------------------Exercise 9.3----------------------------------------------------------------------------------------
\section*{Exercise 9.3}
\par From the data, we see that 
\begin{equation*}
    \overline{X} = \frac{1}{200}(0\cdot 109 + 1\cdot 65 + 2\cdot 22 + 3\cdot 3 + 4\cdot 1) = \frac{61}{100} = \widehat{k} 
\end{equation*}
from the maximum-likelihood estimator for $k$ of a Poisson distribution.

\par Then we have 
\begin{equation*}
    \begin{split}
        P[X = 0] &= \frac{e^{-\widehat{k}}\widehat{k}^0}{0!} = 0.543\\
        P[X = 1] &= \frac{e^{-\widehat{k}}\widehat{k}^1}{1!} = 0.331\\
        P[X = 2] &= \frac{e^{-\widehat{k}}\widehat{k}^2}{2!} = 0.101\\
        P[X = 3] &= \frac{e^{-\widehat{k}}\widehat{k}^3}{3!} = 0.020\\
        P[X \geq 4] &= 1 - \sum_{i = 0}^3P[X = i] = 0.003\\
    \end{split}
\end{equation*}

\par Hence, we have the following expected frequencies $E_i = np_i$ as 
\begin{table}[H]
    \centering
    \begin{tabular}{cc}
        Number of deaths $X$ & Expected Frequency $E_i$\\
        \toprule
        $0$ & $200\cdot 0.543 = 108.67$ \\
        $1$ & $200\cdot 0.331 = 66.29$ \\
        $2$ & $200\cdot 0.101 = 20.22$ \\
        $3$ & $200\cdot 0.020 = 4.11$ \\
        $4$ & $200\cdot 0.003 = 0.71$ \\
        \bottomrule
    \end{tabular}
\end{table}

\par We see that $E_3, E_4 < 5$, Cochran's Rule is not satisfied. We combine the last two categories:
\begin{table}[H]
    \centering
    \begin{tabular}{cc}
        Number of deaths $X$ & Expected Frequency $E_i$\\
        \toprule
        $0$ & $200\cdot 0.543 = 108.67$ \\
        $1$ & $200\cdot 0.331 = 66.29$ \\
        $2$ & $200\cdot 0.101 = 20.22$ \\
        $3$ & $200\cdot 0.020 = 4.82$ \\
        \bottomrule
    \end{tabular}
\end{table}

\par Again, $E_3<5$ still, hence we combine the last two categories again:
\begin{table}[H]
    \centering
    \begin{tabular}{ccc}
        Number of deaths $X$ & Expected Frequency $E_i$ & Observed Frequency $O_i$ \\
        \toprule
        $0$ & $200\cdot 0.543 = 108.67$ & $109$ \\
        $1$ & $200\cdot 0.331 = 66.29$ & $65$ \\
        $2$ & $200\cdot 0.101 = 25.04$ & $26$ \\
        \bottomrule
    \end{tabular}
\end{table}

\par We then test 
\begin{equation*}
    H_0 : \text{the number of deaths follows a Poisson distribution with parameter }k = 0.61,
\end{equation*}
which is equivalent to the test 
\begin{equation*}
    H_0 : \text{the number of deaths follows a multinomial distribution with parameters }(0.543, 0.331, 0.125)
\end{equation*}

\par For $N = 3$ categories, the statistic
\begin{equation*}
    X^2 = \sum^3_{i = 1}\frac{(O_i - E_i)^2}{E_i}
\end{equation*}
follows a chi-squared distribution with $N - 1 - m = 3 - 1 - 1 = 1$ degree of freedom. Now, 
\begin{equation*}
    X^2 = \frac{(109-108.67)^2}{108.67}+\frac{(65-66.29)^2}{66.29}+\frac{(26-25.04)^2}{25.04} = 0.063.
\end{equation*}

\par By using {\tt Mathematica}, we find that the $P-$value of this test is 
\begin{equation*}
    {\tt 1 - CDF[ChiSquareDistribution[1], 0.063] = 0.80}
\end{equation*}
which indicates that
\begin{equation*}
    0.80 \gg 0.05
\end{equation*}

\par This is a strong evidence that we are hard to reject $H_0$. Therefore, we conclude that there is a strong evidence that 
the number annual deaths by kicks from horses follows a Poisson distribution with $k = 0.61$.

\myqed

\newpage
%------------------------------------------------Exercise 9.4----------------------------------------------------------------------------------------
\section*{Exercise 9.4}
%------------------------------------------------Exercise 9.4 i--------------------------------------------------------------------------------------
\subsection*{(i)}
\par From {\tt Mathematica}, we have 
\begin{figure}[H]
    \centering
    \includegraphics[width = 0.9\linewidth]{Figure/boxplot.png}
    \caption*{Box plot for the scores in the Midterm Exam of Ve401 in Spring 2021}
\end{figure}
and 
\begin{figure}[H]
    \centering
    \includegraphics[width = 0.9\linewidth]{Figure/histogram.png}
    \caption*{Histogram for the scores in the Midterm Exam of Ve401 in Spring 2021}
\end{figure}

\par We see that from the box plot, there are no outliers, and the interquartile are quite small, indicates that the data is concentrate 
around the mean. Furthermore, from the histogram, although the overall shape of it is slightly skewed, but overall it follows a rough shape
of a normal distribution.

\myqed

%------------------------------------------------Exercise 9.4 ii--------------------------------------------------------------------------------------
\subsection*{(ii)}
\par By using {\tt Mathematica}, we have 
\begin{equation*}
    \overline{X} = 17.16, \quad S^2 = 18.83.
\end{equation*}

\myqed

%------------------------------------------------Exercise 9.4 iii------------------------------------------------------------------------------------
\subsection*{(iii)}
\par We first see that {\tt Mathematica} use 
\begin{equation*}
    k = \left\lceil 2\cdot 165^{2/5}\right\rceil = 16
\end{equation*}
categories. Furthermore, since a normal distribution have two parameters, namely
\begin{equation*}
    \mu\text{ and } \sigma,
\end{equation*}
hence the degrees of freedom is given by 
\begin{equation*}
    16 - 1 - 2 = 13.
\end{equation*}

\par Overall, we have 
\begin{equation*}
    \mu = 17.1576, \qquad \sigma = 4.32603
\end{equation*}
with 
\begin{center}
    $\begin{array}{l|ll}
        \text{} & \text{Statistic} & \text{P-Value} \\
        \hline
        \text{Pearson }\chi ^2 & 30.9758 & 0.00339924 \\
    \end{array}$    
\end{center}

\par We see that the $P-$value of this test is $0.0034\ll 0.05$, so we reject $H_0$. Hence, we conclude that the scores in the Midterm 
Exam of Ve401 in Spring 2021 does does not follow a normal distribution with 
\begin{equation*}
    \mu = 17.1576, \qquad \sigma = 4.32603.
\end{equation*}

\myqed

\newpage
%------------------------------------------------Exercise 9.5----------------------------------------------------------------------------------------
\section*{Exercise 9.5}
\par We have 
\begin{table}[H]
    \centering
    \begin{tabular}{c | cccc | c}
        \toprule
        & \multicolumn{4}{c}{No. of Accidents} &  \\ \cline{2-5}
        Age & $0$ & $1$ & $2$ & $\geq 3$ & \\
        \midrule
        $18 - 29$ & $748$ & $74$ & $31$ & $9 $ & $n_{1\cdot} = 862$\\
        $30 - 39$ & $821$ & $60$ & $25$ & $10$ & $n_{2\cdot} = 916$\\
        $40 - 49$ & $786$ & $51$ & $22$ & $6 $ & $n_{3\cdot} = 865$\\
        $50 - 59$ & $720$ & $66$ & $16$ & $5 $ & $n_{4\cdot} = 807$\\
        $\geq 60$ & $672$ & $50$ & $15$ & $7 $ & $n_{5\cdot} = 744$\\
        \midrule
        & $n_{\cdot 1} = 3747$ & $n_{\cdot 2} = 301$ & $n_{\cdot 3} = 109$ & $n_{\cdot 4} = 37$ & $n = 4194$\\
    \end{tabular}
\end{table}

\par Suppose that they are independent, namely the age of the driver does not influence the number of accidents. Then we have 
\begin{equation*}
    p_{ij} = p_{i\cdot}p_{\cdot j}, \qquad 1\leq i\leq 5, 1\leq j\leq 4
\end{equation*}

\par Hence, we set up our null hypothesis as 
\begin{equation*}
    H_0 : p_{ij} = p_{i\cdot}p_{\cdot j}, \qquad 1\leq i\leq 5, 1\leq j\leq 4
\end{equation*}

\par Noting that the expected number of elements in $(i,j)^{th}$ cell is 
\begin{equation*}
    E_{ij} = n\cdot \widehat{p_{ij}} = \frac{n_{i\cdot}n_{\cdot j}}{n}
\end{equation*}

\par We then have 
\begin{table}[H]
    \centering
    \begin{tabular}{c | cccc}
        \toprule
         & \multicolumn{4}{c}{Expected no. of Accidents}\\ \cline{2-5}
        Age & $0$ & $1$ & $2$ & $\geq 3$\\
        \midrule
        $18 - 29$ & $770.13$ & $61.87$ & $22.40$ & $7.60$\\
        $30 - 39$ & $818.37$ & $65.74$ & $23.81$ & $8.08$\\
        $40 - 49$ & $772.81$ & $62.08$ & $22.48$ & $7.63$\\
        $50 - 59$ & $720.99$ & $57.92$ & $20.97$ & $7.12$\\
        $\geq 60$ & $664.70$ & $53.39$ & $19.34$ & $6.56$\\
        \midrule
    \end{tabular}
\end{table}

\par Then, we can obtain the statistic
\begin{equation*}
    X^2_{(5-1)(4-1)} = X^2_{12} = \sum^5_{i = 1}\sum^4_{j = 1}\frac{(O_{ij} - E_{ij})^2}{E_{ij}} = 14.40
\end{equation*}

\par For $95\%$ level of confidence, we have 
\begin{equation*}
    \chi^2_{0.95, 12} = 21.03 > X^2_{12}.
\end{equation*}
Therefore, we have no evidence to reject $H_0$, namely we conclude that the age of the driver will not influence the number 
of accidents, and they are independent.

\myqed

\newpage
\appendix
\section*{Mathematica Source Code}
\begin{figure}[H]
    \centering
    \includegraphics[width = 1.1\linewidth]{Mathematica/0001.jpg}
\end{figure}

\begin{figure}[H]
    \centering
    \includegraphics[width = 1.1\linewidth]{Mathematica/0002.jpg}
\end{figure}

\begin{figure}[H]
    \centering
    \includegraphics[width = 1.1\linewidth]{Mathematica/0003.jpg}
\end{figure}

\begin{figure}[H]
    \centering
    \includegraphics[width = 1.1\linewidth]{Mathematica/0004.jpg}
\end{figure}

\begin{figure}[H]
    \centering
    \includegraphics[width = 1.1\linewidth]{Mathematica/0005.jpg}
\end{figure}

\begin{figure}[H]
    \centering
    \includegraphics[width = 1.1\linewidth]{Mathematica/0006.jpg}
\end{figure}

\begin{figure}[H]
    \centering
    \includegraphics[width = 1.1\linewidth]{Mathematica/0007.jpg}
\end{figure}

\begin{figure}[H]
    \centering
    \includegraphics[width = 1.1\linewidth]{Mathematica/0008.jpg}
\end{figure}

\begin{figure}[H]
    \centering
    \includegraphics[width = 1.1\linewidth]{Mathematica/0009.jpg}
\end{figure}

\begin{figure}[H]
    \centering
    \includegraphics[width = 1.1\linewidth]{Mathematica/0010.jpg}
\end{figure}

\end{document}

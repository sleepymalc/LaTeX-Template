\usepackage[T1]{fontenc}
\usepackage[size=custom,width=120,height=72,scale=1.5]{beamerposter}
\usetheme{gemini}
\usecolortheme{gemini}
\usepackage{graphicx}
\usepackage{booktabs}
\usepackage{caption}
\usepackage{cleveref}
\Crefname{equation}{Eq.\!}{Eqs.\!}
\usepackage[backend=biber,style=numeric,defernumbers=true]{biblatex}
\usepackage{svg}
\usepackage{multicol}
\usepackage{mathtools}
\usepackage{subcaption}
\usepackage{etoolbox}
\usepackage{tcolorbox}
\usepackage{xparse}
\usepackage{xcolor}

% figure support
\usepackage{import}
\usepackage{xifthen}
\usepackage{pdfpages}
\usepackage{transparent}
\newcommand{\incfig}[1]{%
	\def\svgwidth{\columnwidth}
	\import{./Figures/}{#1.pdf_tex}
}

\DeclareMathOperator*{\argmax}{arg\,max}
\DeclareMathOperator*{\argmin}{arg\,min}
\newcommand{\at}[3]{\left.#1\right\vert_{#2}^{#3}}
\let\implies\Rightarrow

\DeclareNameFormat{family}{%
	\usebibmacro{name:family}
	{\namepartfamily}
	{\namepartgiven}
	{\namepartprefix}
	{\namepartsuffix}%
	\usebibmacro{name:andothers}}

\DeclareFieldFormat*{cite title}{#1}
\DeclareFieldFormat*{title}{#1}
\renewbibmacro{in:}{}

\DeclareNameAlias{default}{family}
\DeclareNameAlias{sortname}{default}
\DeclareNameAlias{labelname}{default}

\AtEveryBibitem{%
	\clearfield{journal}%
	\clearfield{booktitle}%
	\clearfield{pages}%
}
\renewcommand*{\bibfont}{\footnotesize}

% Undefine existing environments
\let\definition\relax
\let\enddefinition\relax
\let\theorem\relax
\let\endtheorem\relax

% Define `definition` environment
\NewDocumentEnvironment{definition}{o}
{
	\begin{tcolorbox}[colback=GreenYellow!5!white, colframe=GreenYellow!75!black]
		{\color{purple}\bfseries Definition}%
		\IfValueT{#1}{\textnormal{~(#1)}}%
		\textbf{.}~
		}
		{
	\end{tcolorbox}
}

% Define `theorem` environment
\NewDocumentEnvironment{theorem}{o}
{
	\begin{tcolorbox}[colback=SkyBlue!5!white, colframe=SkyBlue!75!black]
		{\color{purple}\bfseries Theorem}%
		\IfValueT{#1}{\textnormal{~(#1)}}%
		\textbf{.}~
		}
		{
	\end{tcolorbox}
}

\NewDocumentEnvironment{highlight}{o}{\begin{tcolorbox}[colback=GreenYellow!5!white,colframe=GreenYellow!75!black]\IfValueT{#1}{{\color{purple}\textbf{#1.}}~}}{\end{tcolorbox}}

% If you have N columns, choose \sepwidth and \colwidth such that
% (N+1)*\sepwidth + N*\colwidth = \paperwidth
\newlength{\sepwidth}
\newlength{\colwidth}
\setlength{\sepwidth}{0.01\paperwidth}
\setlength{\colwidth}{0.31\paperwidth}